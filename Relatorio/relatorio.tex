\documentclass[a4paper]{article}

\usepackage[utf8]{inputenc}
\usepackage[portuges]{babel}
\usepackage{a4wide}
\usepackage{underscore}

\title{Projeto de Programação Orientada aos Objetos\\Grupo }
\author{Diogo Miguel Alves Rocha (A79751)\and Ricardo Milhazes Veloso (A81919) \and Ricardo Jorge Silva Ferreira (82568)}
\date{\today}
\begin{document}

\maketitle

\begin{abstract}
 
 Neste relatório faremos a análise ao nosso projeto de Programação Orientada aos Objetos, no qual consiste em criar um programa em Java, chamado JavaFatura com o objetivo de criar um sistema de gestão e dedução de faturas que permita quer a clientes individuais quer a empresas gerir o seu estado fiscal em termos de faturação.

 \end{abstract}

\tableofcontents

\section{Introdução}
\label{sec:intro} 

O principal objetivo deste projeto foi desenvolver um programa capaz de registar faturas com a finalidade, de as empresas e os clientes individuais terem acesso às mesmas quer a empresa do montante total de vendas e da dedução a que está sujeita , quer o cliente individual ter acesso ao montante total de gastos e saber qual o montante dedutivel. Para isso será necessário um registo no programa e outras funcionalidades que iremos demonstrar mais à frente neste relatório.


\section{Descrição do Problema}
Neste projeto o programa deve ser capaz de gerir o estado fiscal do utilizador para isso deve ser capaz de:

\begin{itemize}
\item Registar um utilizador (Individual ou Empresa);
\item Implementar um login;
\item Criar uma Fatura quer seja uma despesa (C.individual) ou uma venda (C.empresa);
\item Ser possível ao contribuinte individual verificar as faturas emitidas em seu nome , e verificar a respetiva dedução fiscal acumulada;
\item Associar classificação de actividade económica a um documento de despesa;
\item Corrigir a classificação de actividade económica de um documento de despesa;
\item Obter a listagem das facturas de uma determinada empresa, ordenada por data de emissão
ou por valor;
\item Obter por parte das empresas, as listagens das facturas por contribuinte individual num determinado
intervalo de datas;
\item Indicar o total facturado por uma empresa num determinado período;
\item Determinar a relação dos 10 contribuintes que mais gastam em todo o sistema;
\item Determinar as X empresas que mais facturas emitem em todo o sistema e o montante de
deduções fiscais que as despesas registadas (dessas empresas) representam;
\item Gravar o estado da aplicação em ficheiro, para que seja possível retomar mais tarde a execução

\end{itemize}

\section{Concepção da Solução}

A nossa solução foi implementada baseada em diferentes classes , as quais vamos agora especificar:

\begin{itemize}
	\item{}

\end{itemize}

\section{Instruções de utilização}
Em seguida uma espécie de manual de instruções para a utilização do nosso programa:

\begin{itemize}
	\item{Criar um utilizador}
		\begin{itemize}
			\item{Contribuinte individual}
			\item{Contribuinte empresa}
		\end{itemize}
	\item{Contribuinte individual}
		\begin{itemize}
			\item{Fazer login no sistema}
			\item{Verificar as depesas emitidas em seu nome e do agregado familiar}
			\item{Corrigir a classificação de atividade económica}
		\end{itemize}
	\item{Contribuinte empresa}
		\begin{itemize}
			\item{Fazer login no sistema}
			\item{Registar uma factura}
			\item{Acesso às faturas por si emitidas}
		\end{itemize}

\end{itemize}

\end{document}